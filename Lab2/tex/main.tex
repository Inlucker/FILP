\chapter*{Задание 1}
Составить диаграмму вычисления следующих выражений:

(equal 3 (abs - 3))

(equal (+ 1 2) 3) 

(equal (* 4 7) 21)

(equal (* 2 3) (+ 7 2))

(equal (- 7 3) (* 3 2))

(equal (abs (- 2 4)) 3))

%\newline
2. Написать функцию, вычисляющую гипотенузу прямоугольного
треугольника по заданным катетам и составить диаграмму её вычисления.

\begin{lstlisting}[language=LISP, caption=Задание 2]
(defun get_gip(a b)
	(sqrt (+ (* a a) (* b b))))
\end{lstlisting}

%\newline
3. Написать функцию, вычисляющую объем параллелепипеда по 3-м его сторонам, и
составить диаграмму ее вычисления.

\begin{lstlisting}[language=LISP, caption=Задание 3]
(defun get_vol(a b c)
	(* a b c))
\end{lstlisting}

%\newline
4. Каковы результаты вычисления следующих выражений?(объяснить возможную ошибку и
варианты ее устранения)

(list 'a c) - (list 'a 'c)

(cons 'a (b c)) - (cons 'a '(b c))

(cons 'a '(b c))

(caddy (1 2 3 4 5)) - (caddr '(1 2 3 4 5))

(cons 'a 'b 'c) - (cons 'a '(b c))

(list 'a (b c)) - (list 'a '(b c)) - (list 'a 'b 'c)

(list a '(b c)) - (list 'a '(b c))

(list (+ 1 '(length '(1 2 3)))) - (list (+ 1 (length '(1 2 3))))

%\newline
5. Написать функцию longer\_then от двух списков-аргументов, которая возвращает Т, если
первый аргумент имеет большую длину.

\begin{lstlisting}[language=LISP, caption=Задание 5]
(defun longer_than(a b)
	(if (> (length a) (length b)) T nil))
\end{lstlisting}

%\newline
6. Каковы результаты вычисления следующих выражений?

(cons 3 (list 5 6)) = (3 5 6)

(cons 3 '(list 5 6)) = (3 LIST 5 6)

(list 3 'from 9 'lives (- 9 3)) = (3 FROM 9 LIVES 6)

(+ (length '(for 2 too)) (car '(21 22 23))) = 24

(cdr '(cons is short for ans)) = (IS SHORT FOR ANS)

(car (list one two)) = ошибка

(car (list 'one 'two)) = ONE

%\newline
7. Дана функция (defun mystery (x) (list (second x) (first x))).
Какие результаты вычисления следующих выражений?

(mystery (one two)) = ошибка; (mystery '(one two)) = (TWO ONE)

(mystery one 'two)) = ошибка

(mystery (last one two)) = ошибка; (mystery (list 'one 'two)) = (TWO ONE)

(mystery free) = ошибка

%\newline
8. Написать функцию, которая переводит температуру в системе Фаренгейта температуру по Цельсию (defun f-to-c (temp)…).

Формулы: c = 5/9*(f-32); f= 9/5*c+32.0.

Как бы назывался роман Р.Брэдбери "+451 по Фаренгейту" в системе по Цельсию?

\begin{lstlisting}[language=LISP, caption=Задание 8]
(defun f-to-c(temp)
	(* (- temp 32) (/ 5 9)))
\end{lstlisting}

Ответ: 2095/9 = 232,(7)

%\newline
9. Что получится при вычисления каждого из выражений?

(list 'cons t NIL) = (CONS T NIL)

(eval (list 'cons t NIL)) = (T.NILL) = (T)

(eval (eval (list 'cons t NIL))) = ошибка (eval (T))